\subsection*{Fundamentals}
\textbf{Definitions}

\underline{1:} A computer program is said to
learn from experience E with respect to some task T
and some performance measure P, if its performance
on T, as measured by P, improves with experience E.

\underline{2:} Set of computer methods that
analyse observation data to automatically detect
patterns, and then use the uncovered patterns to
perform functions on new un-observed data. Examples
of functions include : prediction, classification,
clustering and more generally decision making.

\textbf{Two main types:}

\underline{Supervised learning:} the goal is to learn a
mapping from inputs x to outputs y given a set of
example data called the training set.

\underline{Unsupervised learning:} the goal is to discover
interesting structures from inputs x given a set of data
called the training set.

A classification task maps inputs x to a finite set of
discrete outputs y. The outputs are the class labels
corresponding to the different categories we want to
predict. A regression task maps inputs x to an infinite set of
continuous outputs y. The outputs are numeric values
corresponding to the variable we want to predict.
