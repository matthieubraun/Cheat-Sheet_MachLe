\subsection*{K-NN}
k-nearest neighbour algorithm is a
method that classifies unlabelled examples based on
their similarity with examples in the training set.
It can be used for both classification and regression.

\textbf{Distance metric:}

\underline{L1 (Manhattan):} $d(I_1, I_2) = \sum_{i=1}^{n} |I_1^p - I_2^p|$

\underline{L2 (Euclidean):} $d(I_1, I_2) = \sqrt{\sum_{i=1}^{n} (I_1^p - I_2^p)^2}$

\textbf{Hyperparameters to tune:}

\underline{Normalization:} type : non, min-max, z-score.

\underline{distance metric:} L1, L2.

\underline{k:} number of neighbours.

K-NN is heavy to compute in terms of memory and CPU time. Memory : full training set needs to be stored.
CPU : the distance is computed against all training examples.
