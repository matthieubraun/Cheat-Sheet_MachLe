\subsection*{Decision Trees}
Implements a sequence of decisions
on individual features allowing to classify the input data
at the end of the sequence. The decisions are taken
comparing a selected feature against a reference value.
The decisions are organised in a binary tree where each
decision split the input space into two branches.

\textbf{Gini impurity:}

\underline{Of a set:}

$G_{set} = 1 - \sum_{i=1}^{K} P(C_k)^2$ where
$P(C_k)=\sfrac{N_k}{N}$ measure how ofen a randomly chosen element
from the set would be incorrectly labeled if it was randomly labeled.
$N_k$ is the number of elements of class $k$ in the set.

\underline{Of a split:}

$G_{split} = \frac{N_{left}}{N}G_{left} + \frac{N_{right}}{N}G_{right}$.
is the weighted sum of the Gini impurity of the two sets after the split.
The weights are the proportions in the set.

\underline{Gain:} $G_{gain} = G_{set} - G_{split}$.

\textbf{Overfitting:}

\underline{Early stopping:} stop growing the tree after a certain depth.

\underline{Pruning:} grow a full tree and trim it afterwards.

\underline{Boosting:} build aditional trees on the weakneses of the previous ones.

\underline{Random Forest:} build multiple trees on random subsets of the data and average
their predictions.

\textbf{XGBoost} : (go-to solution) is a boosting algorithm that uses gradient boosting
to build the trees (implies the existence of a loss function).
