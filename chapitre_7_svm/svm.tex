\subsection*{SVM}
\textbf{Linear SVM:} tries to find the hyperplane that separates the 2
classes and that maximizes the margin between the 2 classes.
SVM are particularly efficient for tasks with a high number of features and a low number of samples.

\underline{Hypothesis function:}

$h_w(\overrightarrow{x})=\text{sign}(b+\overrightarrow{w}^T\overrightarrow{x})$.

\underline{Margin:} distance between the hyperplane and the closest point of each class. Training samples
on the margin boundaries are called support vectors.

\underline{Hinge loss function:} used to calculate the slack :
$\xi_i=H(d(x_i, w, b))$, $d()$ is the distance from the boundary. That means
$\xi_i=0$ if $x_i$ is correctly classified and $\xi_i>0$ if $x_i$ is misclassified or
in the margin.

\underline{Loss function:} for two classes:

$J(w)=C[\frac{1}{N}\sum_{i=1}^{N}y_iH_{C1}(d(x_i, w, b))+(1-y_i)H_{C0}(d(x_i, w, b))]+\sfrac{\left\|w\right\|^2}{2}$.
The smaller the $C$, the greater the number of misclassified points.

\textbf{Kernel functions for non-linear problems:}
Compute a new test sample $x_t$ using a kernel funciton.

\underline{Linear:} $K(x_i, x_j)=x_ix_j$

\underline{Polynomial:} $K(x_i, x_j)=(x_ix_j+1)^d$

\underline{RBF:} $K(x_i, x_j)=\exp(-\gamma\left\|x_i-x_j\right\|^2)$

\underline{Hyperbolic tangent:}

$K(x_i, x_j)=\tanh(\kappa x_ix_j+\delta)$

\textbf{Multi-class classification:}

\underline{One-vs-all:} the SVM with the highest output value assigns the class to a sample.

\underline{One-vs-one:} each SVM assigns a sample to one of the two classes.
The class with the highest number of votes is assigned to the sample.





