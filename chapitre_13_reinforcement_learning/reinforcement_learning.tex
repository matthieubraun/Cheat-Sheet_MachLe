\subsection*{Reinforcement Learning}
Learns from interaction with an environment, optimizing a reward function.
RL problems can be modeled as Markov Decision Processes (MDPs), which consist of a set
of states, a set of actions, a transition function, and a reward function. The goal
of RL is to find a policy that maps states to actions, and maximizes the expected return,
which is the discounted sum of future rewards.

\textbf{Value functions:} estimate the expected return from a given state or
state-action pair. Value functions can be learned by dynamic programming,
\textit{Monte Carlo} or \textit{temporal-difference learning}.

\textbf{Value-based methods (SARSA):} learn a value function and derive a policy from it.

\textbf{Policy-based methods:} methods that learn a policy directly, without using a value function.

\textbf{Exploration-exploitation trade-off:} the dilemma of choosing between actions
that have high expected reward (exploitation) and actions that have high
uncertainty (exploration).

\underline{$\varepsilon$-greedy:} with probability $\varepsilon$, choose a random action, otherwise choose the best action.

\underline{Boltzmann:} $P(a|s)=\frac{e^{\frac{Q(s, a)}{\tau}}}{\sum_{b\in a}e^{\frac{Q(s, b)}{\tau}}}$
Explore based on the probability of each action.

