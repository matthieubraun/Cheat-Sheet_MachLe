\subsection*{K-MEANS}
K-Means is an unsupervised algorithm that finds clusters in the data. It is mathematically
guaranteed to converge but can converge to a local minimum.

\textbf{Terminology:}

\underline{Centroids:} the center of the clusters.

\underline{Codebook:} the set of centroids.

\underline{Partition:} the set of samples assigned to a centroid.

\textbf{Distortion function:}

$J(c, \mu)=\sum_{i=1}^{N} = d(x_n, \mu_{C_n})^2$.

$C_n$ is the centroid of the cluster to which $x_n$ is assigned.
$\mu_{C_n}$ is the centroid of the cluster $C_n$. $d()$ is the distance function and can
be the euclidean or manhattan distance.

\textbf{Elbow method:}
finds the optimal number of clusters. It plots the distortion
function vs the number of clusters. The number of clusters after which the distortion function
starts to decrease slowly is the optimal.

