\subsection*{Convolutional Neural Networks}
Convolutional Neural Networks (CNNs) process spatial data using
convolutions, extracting diverse features from images with multiple
filters. They detect hierarchical features by linking filter layers,
and use new activation functions to mitigate vanishing gradients.
CNNs summarize lower layer feature statistics via down-sampling and
employ techniques like dropout to prevent overfitting.

Each hidden neuron process a small region of the image. Different kernel sizes
allows the identification of features at different scales. Zeros padding allows
the output to have the same size as the input.

Multiple convolutions look for different features. The output of a convolutional
layer with 30 kernels gives 30 images.

Sigmoid activation functions can cause vanishing gradients due to their derivatives
nearing zero. Softmax in the output layer calculates a categorical
probability distribution : $f(x_i)=\sfrac{e^{x_i}}{\sum_{j=1}^{K}e^{x_j}}$.

Maxpooling after convolution reduces computation by down-sampling non-maximal
values, summarizing lower layer feature statistics. It results in smaller images.

The resulting architecture of a CNN is a stack of alternating convolutional and
pooling layers, followed by a fully connected layer.

Dropout in CNNs randomly deactivates neurons during training to prevent
overfitting, enhancing model generalization by reducing complex
co-dependencies on training data.